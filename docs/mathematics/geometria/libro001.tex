\documentclass[a4paper,10pt]{book}
\usepackage[utf8]{inputenc}
\usepackage{amsmath}
\usepackage{hyperref}

%opening
\title{Geometria analitica di Silvio Greco e Paolo Valabrega}
\author{\href{http://www.baudo.hol.es}{giuseppe baudo}}

\begin{document}

\maketitle

\chapter{Elementi di geometria analitica del piano}

Contenuto del capitolo.
\begin{itemize}
 \item Coordinate cartesiane sulla retta e nel piano.
 \item Rappresentazione analitica di figure geometriche come: rette, circonferenze, coniche.
 \item Risoluzione analitica di problemi geometrici elementari.
 \item Cambiamenti di coordiante.
 \item Coordinate polari.
\end{itemize}

Scopi del capitolo. Questo capitolo deve mettere lo studente in grado di rappresentare analiticamente rette, circonferenze e coniche
e di risolvere analiticamente i problemi elementari relativi a questi enti.


\section{APPROFONDIMENTI}
\begin{itemize}
 \item \url{https://sol.unibo.it/SebinaOpac/Opac?action=search&thNomeDocumento=UBO2453492T}
\end{itemize}

\end{document}
