\documentclass[a4paper,10pt]{article}
\usepackage[utf8]{inputenc}
\usepackage{amsmath}
\usepackage{hyperref}

%opening
\title{Sistema di riferimento cartesiano}
\author{\href{http://www.baudo.hol.es}{giuseppe baudo}}

\begin{document}

\maketitle

\section{DEFINIZIONE}
Un sistema di riferimento cartesiano è un sistema di riferimento formato da n rette ortogonali, intersecantesi tutte in un punto
chiamato origine, su ciascuna delle quali si fissa un orientamento (sono quindi dette orientate) e per le quali si fissa
anche un'unità di misura (cioè si fissa una metrica di solito euclidea) che consente di identificare qualsiasi punto dell'insieme
mediante n numeri reali. In questo caso si dice che i punti di questo insieme sono in uno spazio di dimensione n.

\section{NOTE}


\section{ESEMPIO}

\section{APPROFONDIMENTI}
\begin{itemize}
 \item \href{https://it.wikipedia.org/wiki/Sistema_di_riferimento_cartesiano}{https://it.wikipedia.org/wiki/Sistema_di_riferimento_cartesiano}
\end{itemize}

\end{document}
