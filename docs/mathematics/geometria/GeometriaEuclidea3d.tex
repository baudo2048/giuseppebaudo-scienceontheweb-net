\documentclass[a4paper,10pt]{article}
\usepackage[utf8]{inputenc}
\usepackage{amsmath}
\usepackage{hyperref}

%opening
\title{Spazio tridimensionale della geometria euclidea, geometria euclidea dello spazio, spazio euclideo}
\author{\href{http://www.baudo.hol.es}{giuseppe baudo}}

\begin{document}

\maketitle

\section{DEFINIZIONE}
Uno spazio euclideo è uno spazio affine in cui valgono gli assiomi e i postulati della geometria euclidea.

\section{NOTAZIONE}

\section{ESEMPIO}

\section{APPROFONDIMENTI}
\begin{itemize}
 \item \href{https://it.wikipedia.org/wiki/Spazio_euclideo}{https://it.wikipedia.org/wiki/Spazio_euclideo}
 \item \href{https://www.britannica.com/topic/Euclidean-space}{https://www.britannica.com/topic/Euclidean-space}
 \item \href{http://www.molwick.com/it/relativita/324-geometria-spazio.html}{http://www.molwick.com/it/relativita/324-geometria-spazio.html}
 \item \href{http://www.dmmm.uniroma1.it/~giuseppe.accascina/Tesi_di_Laurea/2005-Piselli-Geometria_euclidea_dello_spazio/2005-Piselli-Tesi.pdf}{http://www.dmmm.uniroma1.it/~giuseppe.accascina/Tesi_di_Laurea/2005-Piselli-Geometria_euclidea_dello_spazio/2005-Piselli-Tesi.pdf}
 \item \href{http://www.treccani.it/enciclopedia/tag/spazio-tridimensionale-euclideo/}{http://www.treccani.it/enciclopedia/tag/spazio-tridimensionale-euclideo/}
\end{itemize}

\end{document}
