\documentclass[a4paper,10pt]{article}
\usepackage[utf8]{inputenc}
\usepackage{amsmath}
\usepackage{hyperref}

%opening
\title{Piano passante per un punto e ortogonale ad un vettore}
\author{\href{http://www.baudo.hol.es}{giuseppe baudo}}

\begin{document}

\maketitle

\section{DEFINIZIONE - ASSIOMA CHE DIVENTA DEFINIZIONE??!!!}
Sia $\pi$ il piano passante per un punto $P_0$ ortogonale ad un vettore $n \ne o$. Allora $\pi$ è il luogo
dei punti $P$ dello spazio tali che il vettore $P_0P$ è ortogonale al vettore $n$, ovvero:
\[
 \pi = \{ P \in S_3 | P_0P \cdot n = o  \} 
\]


\section{NOTA}
Qui stiamo facendo una cosa di estrema importanza stiamo formulando in linguaggio matematico moderno quello che per Euclide era un
concetto primitivo non dimostrabile (e non definibile! in un certo senso).

Vorrei capire però se $P_0P \cdot n$ è il prodotto scalare o vettoriale?

\section{ESEMPIO}

\section{APPROFONDIMENTI}
\begin{itemize}
 \item http://calvino.polito.it/~salamon/P/G/alga11.pdf
\end{itemize}

\end{document}
