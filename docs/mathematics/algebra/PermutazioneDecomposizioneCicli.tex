\documentclass[a4paper,10pt]{article}
\usepackage[utf8]{inputenc}
\usepackage{amsmath}
\usepackage{hyperref}

%opening
\title{Decomposizione in cicli disgiunti}
\author{www.baudo.hol.es}

\begin{document}

\maketitle

\section{DEFINIZIONE}
Decomporre una permutazione in cicli disgiunti vuol dire rappresentarla sotto forma di cicli.

\section{ESEMPIO}
Come fare a ottenere una rappresentazione in cicli di una permutazione? Basta "seguire" una lettera qualunque fino a trovare
un ciclo: per esempio, in 
\[
\left( \begin{array}{cccc} 1 & 2 & 3 & 4 \\ 3 & 1 & 2 & 4 \\ \end{array} \right) 
\]
abbiamo che 1 va in 3, 3 va in 2 e 2 va in 1; quindi il primo ciclo che troviamo è (123). A questo punto non ci rimane che 4,
che però va in sé, e formerebbe un ciclo di lunghezza 1. I cicli di lunghezza 1 per convenzione non si scrivono, e
quindi la permutazione si scrive (123).

NB: Secondo me se segui questo procedimento per forza di cose devi trovare cicli disgiunti.

\section{APPROFONDIMENTI}
\begin{itemize}
 \item DISPENSA: Permutazioni \cite{permutazione2}
 \item DISPENSA: Orbite e cicli di una permutazione \cite{permutazione4}
 \item DISPENSA: Lezione 9 \href{http://www.science.unitn.it/~luminati/didattica/md/1998/diario/Lezione_9.htm}{http://www.science.unitn.it/~luminati/didattica/md/1998/diario/Lezione_9.htm}
 \item ESERCIZI SVOLTI: Algebra 1 \href{http://www.mat.uniroma3.it/users/gabelli/AL1_06_07/soluzioni2esonero.pdf}{http://www.mat.uniroma3.it/users/gabelli/AL1_06_07/soluzioni2esonero.pdf}
\end{itemize}

\bibliography{AlgebraIndex}
\bibliographystyle{plain}
\end{document}
