\documentclass[a4paper,10pt]{article}
\usepackage[utf8]{inputenc}
\usepackage{amsmath}
\usepackage{hyperref}
\usepackage[english]{babel}
\usepackage{natbib}

%opening
\title{Principio Inclusione - Esclusione}
\author{baudo81[at]gmail.com}

\begin{document}

\maketitle

\section{DEFINIZIONE}
Il principio di inclusione-esclusione è un'identità che mette in relazione la cardinalità di un insieme, espresso come unione di insiemi finiti,
con le cardinalità di instersezioni tra questi insiemi.

\section{HISTORY}
IL principio è stato utilizzato da Nicolaus II Bernoulli (1695-1726); la formula viene attribuita ad Abraham de Moivre (1667-1754);
per il suo utilizzo e per la comprensione della sua portata vengono ricordati Joseph Sylvester (1814-1897) ed Henri Poincaré (1854-1912). 

\section{APPROFONDIMENTI}
\begin{itemize}
 \item PAPER \cite{pinclescl}
\end{itemize}

\bibliography{AlgebraIndex}
\bibliographystyle{alpha}
\end{document}
