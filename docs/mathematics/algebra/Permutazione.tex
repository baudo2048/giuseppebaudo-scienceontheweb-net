\documentclass[a4paper,10pt]{article}
\usepackage[utf8]{inputenc}
\usepackage{amsmath}
\usepackage{hyperref}

%opening
\title{Permutazione}
\author{www.baudo.hol.es}

\begin{document}

\maketitle

\section{DEFINIZIONE}
Sia $X$ un insieme non vuoto. Si dice permutazione su $X$ ogni applicazione bigettiva di $X$ in se stesso. \cite{permutazione1}

\section{NOTAZIONE - FORMA MATRICIALE}
In generale, per indicare una permutazione si usano le lettere greche minuscole, es. $\sigma$, e la cosiddetta notazione matriciale,
nella quale sono riportarte (nella seconda riga) le immagini secondo $\sigma$ degli elementi di $X$ (scritti nella prima riga):
\[
\left(
 \begin{array}{cccc}
  1 & 2 & ... & n \\
  \sigma(1) & \sigma(2) & ... & \sigma(n)
 \end{array}
\right)
\] \cite{permutazione1}

\section{NOTAZIONE (2) - CICLI}
see \href{./PermutazioneCicli.html}{Ciclo di una permutazione}

\section{PERMUTAZIONE IDENTICA - ELEMENTO NEUTRO RISPETTO ALLA COMPOSIZIONE DI PERMUTAZIONI}
In questa notazione, l'applicazione identica corrisponde ad una matrice con due righe uguali:
\[
\left(
 \begin{array}{cccc}
  1 & 2 & ... & n \\
  1 & 2 & ... & n \\
 \end{array}
\right)
\]

Indicheremo tale applicazione (detta permutazione identica), più semplicemente, con il simbolo $id$. \cite{permutazione1}

\section{INVERSA DI UNA PERMUTAZIONE}
per ottenere l’inversa di una permutazione basta scambiare la prima e la seconda riga e riordinare la prima. \cite{permutazione2}

\section{INSIEME DELLE PERMUTAZIONI}
Denoteremo con $S(X)$ l'insieme delle permutazioni su $X$. \cite{permutazione1}

Il numero di elementi di $S(X)$ è uguale a $n!$, dove $n$ è il numero di elementi dell'insieme $X$.

\section{NOTE}
Una permutazione è una funzione

\section{APPROFONDIMENTI}
\begin{itemize}
 \item \href{./pdf/PERMUTAZIONE/lezione4.pdf}{DISPENSA: Gruppi di permutazioni} \cite{permutazione1}
 \item \href{./pdf/PERMUTAZIONE/permutazioni.pdf}{DISPENSA: Permutazioni} \cite{permutazione2}
 \item \href{http://www.dm.uniba.it/~barile/Rete4/algebra1_pdf/lezione18.pdf}{DISPENSA: Orbite e cicli di una permutazione.} \cite{permutazione4}
 \item \href{http://progettomatematica.dm.unibo.it/Permutazioni/homepg.htm}{PROGETTO MATEMATICA: Permutazioni} \cite{permutazione3}
\end{itemize}


\bibliography{AlgebraIndex}
\bibliographystyle{plain}
\end{document}
