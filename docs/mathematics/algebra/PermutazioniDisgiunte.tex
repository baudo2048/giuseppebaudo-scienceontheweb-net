\documentclass[a4paper,10pt]{article}
\usepackage[utf8]{inputenc}
\usepackage{amsmath}
\usepackage{hyperref}

%opening
\title{Permutazioni disgiunte}
\author{www.baudo.hol.es}

\begin{document}

\maketitle

\section{DEFINIZIONE}
Due permutazioni $\alpha$ e $\beta$ si definiscono disgiunte se gli oggetti che non sono fissi per una permutazione
sono fissi per l'altra, ovvere se:
\[
 (X \setminus F(\alpha))\cap (X \setminus F(\beta))=\o{}
\]


\section{NOTAZIONE}

\section{ESEMPIO 1}
Per esempio, (123) e (45) sono disgiunti, ma (123) e (124) no. 

\section{ESEMPIO 2}
In $S_{8}$, $\alpha = \left( \begin{array}{cccccccc} 1 & 2 & 3 & 4 & 5 & 6 & 7 & 8 \\ 3 & 2 & 4 & 7 & 5 & 6 & 1 & 8 \\ \end{array} \right)$
e $\beta = \left( \begin{array}{cccccccc} 1 & 2 & 3 & 4 & 5 & 6 & 7 & 8 \\ 1 & 8 & 3 & 4 & 5 & 6 & 2 & 8 \\ \end{array} \right)$
sono disgiunte, infatti $\{ 1,3,4,7 \} \cap \{ 2,8 \} = \o{}$

\section{APPROFONDIMENTI}
\begin{itemize}
 \item TESI DI LAUREA: Il gruppo simmetrico $S_{n}$ \cite{simmetrico1}
\end{itemize}

\bibliography{AlgebraIndex}
\bibliographystyle{plain}
\end{document}
