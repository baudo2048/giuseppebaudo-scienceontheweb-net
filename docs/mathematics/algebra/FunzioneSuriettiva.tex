\documentclass[a4paper,10pt]{article}
\usepackage[utf8]{inputenc}
\usepackage{amsmath}
\usepackage{hyperref}
\usepackage[english]{babel}
\usepackage{natbib}

%opening
\title{Funzione suriettiva}
\author{baudo81[at]gmail.com}

\begin{document}

\maketitle

\section{DEFINIZIONE}
Una funzione si dice suriettiva (o surgettiva, o una suriezione) quando ogni elemento del codominio è immagine di almeno un elemento
del dominio. In tal caso si ha che l'immagine coincide con il codominio. \cite{wfunzsur}

\section{APPROFONDIMENTI}
\begin{itemize}
 \item WIKIPEDIA \cite{wfunzsur}
\end{itemize}

\bibliographystyle{plain}
\bibliography{AlgebraIndex}
\end{document}
