\documentclass[a4paper,10pt]{article}
\usepackage[utf8]{inputenc}
\usepackage{amsmath}
\usepackage{hyperref}

%opening
\title{Algebra - Algebra 1 - Algebra 2 - Complementi di algebra - index}
\author{}

\begin{document}

\maketitle



\section{Prerequisites}
\begin{itemize}
 \item Teoria degli insiemi, funzioni, applicazione, prodotto cartesiano.
 \item \href{./CalcProb.html}{Elementi di calcolo delle probabilità}
 \item \href{./AssiomiProb.html}{Assiomatizzazione del calcolo delle probabilità}
 \item Matrici
\end{itemize}

\section{Syllabus}
\begin{itemize}
 \item Permutazioni
 \item \href{./CoefficientiBinomiali.html}{Coefficienti binomiali: definizione e proprietà}
 \item Principio di inclusione-esclusione
 \item Permutazioni senza punti fissi
 \item Gruppo, sottogruppo, gruppo abeliano
 \item Gruppo generale lineare e gruppo simmetrico
 \item Elemento neutro di un gruppo
 \item classi resto modulo n rispetto alla somma
 \item \href{./OmomorfismoGruppi.html}{Omomorfismo di gruppi}
 \item Nucleo, nucleo di una funzione, nucleo di applicazione lineare
 \item \href{./Nucleo.html}{Nucleo di omomorfismo di gruppi}
 \item Insieme degli omomorfismo di gruppi ( dati due gruppi $G$ e $H$ in simboli: $Hom(G,H)$)
\end{itemize}

\section{Esercizi}
\begin{itemize}
 \item \href{./esercizio1.html}{Esercizio 1: Dimostrare che una matrice è sottogruppo di $GL_{n}$}
 \item \href{./esercizio2.html}{Esercizio 2: Dimostrare che una funzione è omomorfismo di gruppi}
 \item \href{./esercizio3.html}{Esercizio 3: Trovare il nucleo di un omomorfismo di gruppi}
\end{itemize}

\end{document}
